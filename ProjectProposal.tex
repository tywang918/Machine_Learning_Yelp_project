\documentclass[11pt,a4paper]{article}
\usepackage[latin1]{inputenc}
\usepackage{amsmath}
\usepackage{amsfonts}
\usepackage{amssymb}
\usepackage{graphicx}
\usepackage{enumerate}
\usepackage{fullpage}
\usepackage{mathrsfs}
\usepackage{longtable}
\usepackage{graphicx}
\usepackage{float}
\usepackage{booktabs}
\author{Gene Burinsky Jianwei Li Tianyi Wang Xiaoxian Wu}
\title{Stat 561: Project Proposal}

\begin{document}
	\maketitle
	
	Our intention is to classify gentrifying hipsters, gentrifying neighborhoods, and make predictions on what area will be gentrified next. 
	\section{Data}
	Although proprietary and technically illegal\footnote{Yelp's employees were condescending and noncooperative in our attempt to purchase the data so that's Gene's justification}, we have scraped data on all businesses in San Diego, CA, as well as, data on almost all of the users who have reviewed restaurants found among the business data. For businesses we have: price category, coordinates, number of reviews, average rating, who reviewed, the text of reviews, the date of the reviews, category, and other miscellaneous information such as phone number or open hours. For users, we have the businesses that they have reviewed (including non-San Diego ones), the number of friends users have, their elite number, and who the users' friends are. Additionally, we intend to overlay/merge Yelp data with either house prices from Zillow or the Census Community Survey (the average rental/housing prices, community ethnic composition, etc) to cross-check our inferences and aid in our learning procedures. 
	
	
	\section{Classifying gentrifying neighborhoods}
	We first classify business data into two categories, whether they are \$ or pricier than \$\$. We then use SVM and a radius neighbor classifier to categorize either a cluster or each business based on the surrounding members. For the radius classifier, we code the number of \$ for the businesses so that cheap businesses are 1 and more expensive ones are >2. If based on its neighbors the business gets a score <2 then it's cheap, if its >3 then it's an expensive business. Expensive businesses classified as a \$ business are misclassified and we assume them to be a gentrifier. There are very well mixed areas; therefore, there will be scores that fall between $[2,3]$ which we will discard. In SVM, there is a similar concept, where well mixed neighborhoods will likely be within the margin so we will disregard those and misclassified businesses outside of the error margin will be classified gentrifiers. 
	
	
	\section{Classifying Users}
	We want to categorize users into 4-5 categories based on some unsupervised clustering algorithm (ie k-means) on non-spatial data. Alternatively, using the businesses that the users have reviewed, we can build use latent variables to look whether some users tend to visit particular areas over others. 
	Apart from using the unsupervised clustering algorithm to categorize users, we are also considering using MCMC with mixture models. To implement this, we plan to build a graphical model with a latent variable z following multinomial (actually 4 or 5-nomial), and some hyperpriors for normals each user's data follows. Then we set up the Gibbs sampler to estimate the parameters. 
	
	Once we have the categories, we aim to check the users who reviewed gentrifying businesses and which category of users they tend to fall in most frequently. 
	
	\section{Goal}
	In the end, our goal is twofold: (a) to predict where in San Diego will more gentrifying businesses open (b) who will go there.
	
\end{document}
